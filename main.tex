\documentclass{programmierpraktikum}
\vorlesung{Programmierpraktikum}
\semester{Sommersemester 2012}
\betreuer{Wilfried Linder}
\subtitle{Bomberman}

\begin{document}

\maketitle
Erstellt werden soll eine Neuimplementierung des bekannten Spiels Bomberman. Bei diesem Spiel
geht es darum, sich mit einer Spielfigur durch ein Labyrinth zu bewegen und durch Verwendung der
Bomben Gegner und Hindernisse zu überwinden.
%
\section{Spielprinzip nach Wikipedia}
In der klassischen Variante besteht das Spielfeld aus einer Anordnung von zerstörbaren und
unzerstörbaren Wänden. Durch das Legen von Bomben können somit immer mehr Bereiche des
Spielfelds erschlossen werden. Hinter einigen Wänden verstecken sich Bonusgegenstände. Die
wichtigsten, die in nahezu jeder Version vorkommen, sind die Bombe, mit ihr erhält der Spieler die
Möglichkeit, eine zusätzliche Bombe zeitgleich zu legen (maximal 8). Eine Flamme erhöht die
Reichweite einer Bombe um ein Feld. Zudem gibt es sogenannte Megabomben, Minen oder
ferngesteuerte Bomben. In manchen Versionen gibt es als Powerup einen Handschuh, der es
Bomberman ermöglicht, Bomben aufzunehmen und zu werfen. Je nach Version kann man entweder
mit einem Boxhandschuh oder mit einem Fuß (Bombenkick-Item) Bomben bis zur nächstgelegenen
Wand wegstoßen.
Die Explosion wird durch Feuerstrahlen in alle vier Richtungen des zweidimensionalen Raums
dargestellt und bringt andere Bomben (falls sie durch die Sprengkraft erreicht werden) zur
sofortigen Zündung, was gewisse Taktiken ermöglicht und erfordert. In den meisten Spielversionen
muss der Spieler sich innerhalb einer bestimmten Zeit zum Ausgang begeben. Dieser ist innerhalb
der zerstörbaren Wände versteckt. Er kann erst betreten werden, nachdem alle Gegner zerstört
wurden.
Ein beliebter Spielmodus von Bomberman ist auch der Mehrspielermodus, in welchem alle
Mitspieler besiegt werden müssen.
%
\section{Mindestanforderungen \& erster Meilenstein}
Für den ersten Meilenstein müssen bereits bestimmte Mindestanforderungen erfüllt werden:
\begin{enumerate}
  \item Darstellung eines Spielfelds in 2D. Dabei müssen zunächst nur feste Mauerstücke implementiert sein, der Ausgang darf dann offen zugänglich sein.
  \item Steuerung einer Figur durch das Spielfeld.
  \item Implementierung einer Bombe mit festem Radius.
  \item Automatisches Neustarten eines Spiels nach Sieg oder Niederlage, Menüführung.
\end{enumerate}
Stichtag für den ersten Meilenstein ist der \textbf{Übungstermin in der Woche vom 14. -- 19. Mai 2012}
\section{Zusatzanforderungen}
Zusätzlich zu diesen Anforderungen wird es im Laufe des Praktikums einen Katalog weiterer Anforderungen geben, von denen einzelne ausgewählt und implementiert werden sollen. Dabei werden auch frei gewählte Erweiterungen erlaubt sein, wobei die Wertung dann in Absprache mit dem Übungsgruppenleiter erfolgt. Es lohnt sich also, bis zum ersten Meilenstein bereits mehr als die Mindestanforderungen zu erfüllen.
Beispiele:
\begin{itemize}
  \item Steuerung einer weiteren Spielfigur (Tastatur, Netzwerk, Computergegner)
  \item Implementierung verschiedener Bomben, Handschuhe, etc
  \item Implementierung weiterer Arten von Mauern
  \item Einlesen der Level aus Dateien
\end{itemize}
\emph{Diese Anforderungen sind Zusatzanforderungen und müssen zum ersten Meilenstein nicht erfüllt sein.}
\section{Formales}
Abgesehen von den inhaltlichen Anforderungen gibt es noch einige formale Anforderungen an das Projekt:
\begin{itemize}
  \item Die Implementierung erfolgt in Java, wenn nicht anders abgesprochen. Wei\-te\-re Pro\-gram\-mier\-sprachen sind erlaubt. Da aber nicht jeder Tutor jede Sprache beherrscht und im Zweifelsfall Support leisten kann, muss eine von Java abweichende Wahl der Programmiersprache mit uns abgesprochen werden.
  \item Die Verwendung von Bibliotheken, die nicht zum Sprachumfang des normalen JDKs gehören, sollte bitte ebenfalls mit uns abgesprochen werden. Ausnahme hiervon ist die Verwendung von JUnit zur Implementierung von Unittests.
  \item Das Projekt ist als Gruppenarbeit gedacht und soll auch als solche durchgeführt werden. Eine Aufteilung der Aufgaben auf die Gruppe ist Pflicht und soll auch mit den Tutoren abgesprochen werden.
  \item Als zentrales Repository für die Versionierung steht uns Github zur Verfügung. Die Funktionen Fork \& Pullrequest (die nicht zum Funktionsumfang von Git an sich gehören) sollen nicht verwendet werden. \textbf{Insbesondere dürfen keine Pull-Requests in das Projekt-Repository akzeptiert werden. Nichtbeachtung führt im schlimmsten Fall zum Ausschluss}
\end{itemize}
\end{document}

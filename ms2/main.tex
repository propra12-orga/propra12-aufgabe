\documentclass{programmierpraktikum}
\vorlesung{Programmierpraktikum}
\semester{Sommersemester 2012}
\betreuer{Wilfried Linder}
\subtitle{Bomberman}

\begin{document}

\maketitle
Zusätzlich zu den Zielen in Meilenstein 1, sollen weitere Ziele erfüllt werden:
%
\section{zweiter Meilenstein}
Wie für den Ersten müssen auch für den zweiten Meilenstein bestimmte Mindestanforderungen erfüllt werden:
\begin{enumerate}
  \item Nachdem die Bombe in Meilenstein 1 noch keine Auswirkungen auf den Spieler hatte, muss sie jetzt funktionsfähig sein
  \item Bomben sollen Kettenreaktionen auslösen können (Bombe wird gezündet durch Explosion einer anderen)
  \item Es soll einen Modus für 2 Spieler geben,
  \item Das Level soll aus einem (programmiersprachenunabhängigem) Dateiformat eingelesen werden können, Empfehlenswert hier wäre beispielsweise XML. Wichtig: Einen Java Objekt zu serialisieren ist NICHT unabhängig von der Progammiersprache,
  \item Der Ausgang soll sich auch hinter Mauerteilen verbergen können,
  \item Es soll eine Dokumentation per Javadoc generiert werden können. Diese soll NICHT hochgeladen werden, sondern wird bei Bedarf von uns generiert. Es reicht, wenn die passenden (JavaDoc-)Kommentare im Quelltext eingefügt sind.
\end{enumerate}
Stichtag für den zweiten Meilenstein ist der \textbf{Übungstermin in der Woche vom 11. -- 15. Juni 2012}
\section{Zusatzanforderungen}
Zusätzlich zu diesen Anforderungen wird es im Laufe des Praktikums einen Katalog weiterer Anforderungen geben, von denen einzelne ausgewählt und implementiert werden sollen. Dabei werden auch frei gewählte Erweiterungen erlaubt sein, wobei die Wertung dann in Absprache mit dem Übungsgruppenleiter erfolgt. Es lohnt sich also, bis zum zweiten Meilenstein bereits mehr als die Mindestanforderungen zu erfüllen.
\section{Formales}
Abgesehen von den inhaltlichen Anforderungen gibt es noch einige formale Anforderungen an das Projekt - diese haben sich bisher nicht geändert:
\begin{itemize}
  \item Die Implementierung erfolgt in Java, wenn nicht anders abgesprochen. Wei\-te\-re Pro\-gram\-mier\-sprachen sind erlaubt. Da aber nicht jeder Tutor jede Sprache beherrscht und im Zweifelsfall Support leisten kann, muss eine von Java abweichende Wahl der Programmiersprache mit uns abgesprochen werden.
  \item Die Verwendung von Bibliotheken, die nicht zum Sprachumfang des normalen JDKs gehören, sollte bitte ebenfalls mit uns abgesprochen werden. Ausnahme hiervon ist die Verwendung von JUnit zur Implementierung von Unittests.
  \item Das Projekt ist als Gruppenarbeit gedacht und soll auch als solche durchgeführt werden. Eine Aufteilung der Aufgaben auf die Gruppe ist Pflicht und soll auch mit den Tutoren abgesprochen werden.
  \item Als zentrales Repository für die Versionierung steht uns Github zur Verfügung. Die Funktionen Fork \& Pullrequest (die nicht zum Funktionsumfang von Git an sich gehören) sollen nicht verwendet werden. \textbf{Insbesondere dürfen keine Pull-Requests in das Projekt-Repository akzeptiert werden. Nichtbeachtung führt im schlimmsten Fall zum Ausschluss}
\end{itemize}
\end{document}

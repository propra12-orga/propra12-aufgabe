\documentclass{programmierpraktikum}
\vorlesung{Programmierpraktikum}
\semester{Sommersemester 2012}
\betreuer{Wilfried Linder}
\subtitle{Bomberman}

\begin{document}

\maketitle
Zusätzlich zu den Zielen in Meilenstein 2, sollen weitere Ziele erfüllt werden:
%
\section{dritter Meilenstein}
Wie für den Zweiten müssen auch für den dritten Meilenstein bestimmte Mindestanforderungen erfüllt werden:
\begin{enumerate}
  \item Zwei Spieler sollen über das Netzwerk miteinander spielen können,
  \item Benutzerhandbuch und Dokumentation müssen vorhanden sein,
  \item Aus der zur Verfügung gestellten Liste müssen Zusatzaufgaben im Wert von 10 Punkten erreicht worden sein und
  \item das Spiel darf nicht mehr als ASCII-Art vorliegen. Eine grafische Ausgabe ist Pflicht!
\end{enumerate}
Stichtag für den dritten Meilenstein ist der \textbf{Übungstermin in der Woche vom 2. -- 6. Juli 2012}
\section{Zusatzanforderungen}
Der dritte Meilenstein schließt die Anforderungen an das Programm ab. Es wird keine weiteren Anforderungen geben.
\section{Formales}
Abgesehen von den inhaltlichen Anforderungen gibt es noch einige formale Anforderungen an das Projekt - diese haben sich bisher nicht geändert:
\begin{itemize}
  \item Die Implementierung erfolgt in Java, wenn nicht anders abgesprochen. Wei\-te\-re Pro\-gram\-mier\-sprachen sind erlaubt. Da aber nicht jeder Tutor jede Sprache beherrscht und im Zweifelsfall Support leisten kann, muss eine von Java abweichende Wahl der Programmiersprache mit uns abgesprochen werden.
  \item Die Verwendung von Bibliotheken, die nicht zum Sprachumfang des normalen JDKs gehören, sollte bitte ebenfalls mit uns abgesprochen werden. Ausnahme hiervon ist die Verwendung von JUnit zur Implementierung von Unittests.
  \item Das Projekt ist als Gruppenarbeit gedacht und soll auch als solche durchgeführt werden. Eine Aufteilung der Aufgaben auf die Gruppe ist Pflicht und soll auch mit den Tutoren abgesprochen werden.
  \item Als zentrales Repository für die Versionierung steht uns Github zur Verfügung. Die Funktionen Fork \& Pullrequest (die nicht zum Funktionsumfang von Git an sich gehören) sollen nicht verwendet werden. \textbf{Insbesondere dürfen keine Pull-Requests in das Projekt-Repository akzeptiert werden. Nichtbeachtung führt im schlimmsten Fall zum Ausschluss}
\end{itemize}
\end{document}
